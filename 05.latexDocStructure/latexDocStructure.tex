% 导言区
\documentclass{ctexart}

% \usepackage{ctex}
% ========设置标题的格式========
\ctexset{
    section = {
        format+ = \zihao{-4} \heiti \raggedright,
        name = {,、},
        number = \chinese{section},
        beforeskip = 1.0ex plus 0.2ex minus .2ex,
        afterskip = 1.0ex plus 0.2ex minus .2ex,
        aftername = \hspace{0pt}
    },
    subsection = {
        format+ = \zihao{5} \heiti \raggedright,
        % name={\thesubsection、},
        name = {,、},
        number = \arabic{subsection},
        beforeskip = 1.0ex plus 0.2ex minus .2ex,
        afterskip = 1.0ex plus 0.2ex minus .2ex,
        aftername = \hspace{0pt}
    }
}

% 正文区
\begin{document}
    \section{引言}
    近年来,随着逆向工程和三维重建技术的发展和应用,获取现实世界中物体的三维数据
    的方法越来越多的关注和研究,很多研究机构和商业公司都陆续推出了自己的三维重建系统。

    近年来,随着逆向工程和三维重建技术的发展和应用,\\获取现实世界中物体的三维数据
    的方法越来越多的关注和研究,\par 很多研究机构和商业公司都陆续推出了自己的三维重建系统。
    \section{实验方法}
    \section{实验结果}
    \subsection{数据}
    \subsection{图表}
    \subsubsection{实验条件}
    \subsubsection{实验过程}
    \subsection{结果分析}
    \section{结论}
    \section{致谢}
\end{document}